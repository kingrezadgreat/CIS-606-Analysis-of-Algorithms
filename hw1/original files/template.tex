\documentclass[11pt]{article}

\setlength{\oddsidemargin}{0in}
\setlength{\textwidth}{6.5in}
\setlength{\topmargin}{-0.5in}
\setlength{\textheight}{8.75in}
\setlength{\parindent}{0pt}
\setlength{\parskip}{6pt}

\usepackage{fancyhdr}
\pagestyle{fancy}
\lhead{HW1}
\rhead{Chip Treewalker}

\usepackage{epsfig,graphicx}

\usepackage{amsmath}

\usepackage{clrscode3e}

\begin{document}

\thispagestyle{plain}

\begin{center}
{\Large \bf CIS 606 \hfil Homework 1 \hfil Fall 2019} \\
\end{center}

\vskip 1in 

\centerline{\includegraphics[width=3in]{Chip.jpg}}

\vskip 0.5in 

\begin{center}
\begin{tabular}{ll}
{\bf Name:}     & {\bf Chip  Treewalker } \\ \\
{\bf Login ID:} & {\bf chtreewa }   
\end{tabular}
\end{center}

\newpage

\begin{enumerate}

\itemsep 0.35in

\item We can sum $\sum_{k=1}^{n} k = \frac{ n ( n + 1)}{2} $ inside text or
      use the displayed math like below:
      
      \[
        \sum_{k=1}^{n} k = 1 + 2 + \cdots + n = \frac{ n ( n + 1)}{2}
      \]

\item The recurrence \eqref{eq:msort} shows the worst-case running time $T(n)$ 
      of mergesort:

%%note that if you don't want eq auto numbering, use equation*

      \begin{equation} 
         T(n)=\begin{cases}
               c             & \text{if $n = 1$},\\
               2T(n/2) + cn  & \text{if $n > 1$}.  \label{eq:msort}
              \end{cases}
      \end{equation}

      Using the master theorem in Chapter 4, we can 
      get $T(n) = \Theta (n \log{} n)$.

\item The recurrence \eqref{eq:bsearch} shows the worst-case running time $T(n)$
      of binary search:

      \begin{equation} 
         T(n)=\begin{cases}
               c             & \text{if $n = 1$},\\
               T(n/2) + c    & \text{if $n > 1$}.  \label{eq:bsearch}
              \end{cases}
      \end{equation}

      Using the master theorem in Chapter 4, we can 
      get $T(n) = \Theta (\log{} n)$.


\item Browse {\tt https://www.cs.dartmouth.edu/\textasciitilde thc/clrscode/clrscode3e.pdf} 
      to learn how to use the clrscode3e package in LaTex  to typeset 
      pseudocode.  

\begin{codebox}
\Procname{$\proc{Insertion-Sort}(A)$}
\li \For $j \gets 2$ \To $\attrib{A}{length}$
\li \Do
        $\id{key} \gets A[j]$
\li \Comment Insert $A[j]$ into the sorted sequence
        $A[1 \twodots j-1]$.
\li     $i \gets j-1$
\li     \While $i > 0$ and $A[i] > \id{key}$
\li     \Do
            $A[i+1] \gets A[i]$
\li         $i \gets i-1$
        \End
\li     $A[i+1] \gets \id{key}$
    \End
\end{codebox}


   
\end{enumerate}

\end{document}

